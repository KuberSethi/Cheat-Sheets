\documentclass{article}

%Insert the title inside the bracketed section below

\title{%
  Question 6.9 from CTCI: \\
  100 Lockers}

\usepackage{listings}
\usepackage{color}
\usepackage{amsmath}

\definecolor{dkgreen}{rgb}{0,0.6,0}
\definecolor{gray}{rgb}{0.5,0.5,0.5}
\definecolor{mauve}{rgb}{0.58,0,0.82}

\lstset{frame=tb,
  language=Java,
  aboveskip=3mm,
  belowskip=3mm,
  showstringspaces=false,
  columns=flexible,
  basicstyle={\small\ttfamily},
  numbers=none,
  numberstyle=\tiny\color{gray},
  keywordstyle=\color{blue},
  commentstyle=\color{dkgreen},
  stringstyle=\color{mauve},
  breaklines=true,
  breakatwhitespace=true,
  tabsize=3
}



\begin{document}


\maketitle 

\section*{Question}

There are one hundred closed lockers in a hallway. A man begins by opening all one hundred lockers. Next, he closes every second locker. Then he goes to every third locker and closes it if it is open or opens it if it is closed (e.g. he toggles every third locker). After his one hundreth pass in the hallway, in which he toggles only locker 100, how many lockers are open? 

\section*{Explanation and Algorithm}

Let's choose an arbitrary locker $n$.

First, let's consider the passes in which we toggle the $n$th locker. We know that we will toggle the nth locker on the first pass (when we open all 100 lockers), and the nth pass (when we open every $n$th locker). However, we must also consider that the $n$th locker will be toggled on the $k$th pass where $k < n$, every $k$th locker is toggled, and $n$ is a multiple of $k$. For example, let us consider when the 35th locker is toggled. This locker will be toggled on the 1st pass (we open all 100 lockers), the 5th pass (toggle every 5th locker), 7th pass (toggle every 7th locker), and the 35th pass (toggle every 35th locker). It is also worth noting that after the nth pass, the $n$th locker will no longer be toggled. This is because in all the subsequent passes, we will be toggling lockers that are a multiple of the pass number.

Next, we must consider what this tells us about the state of the lockers. After the nth pass, will the $n$th locker be open or closed?  In the example of the 35th locker, we know that it is toggled on the 1st, 5th, 7th, and 35th passes. So it is open on the 1st pass, closed on the 5th pass, opened on the 7th pass, and closed on the 35th pass. Thus, the 35th locker will ultimately remain closed. We can take any other locker with an even number of factors and we will end up with the same result. The locker will be opened and closed an even number of times so it will ultimately remain closed. By the same thought process, we can determine that a locker will remain open if it has an odd number of factors. To figure out how many lockers will be left open after the 100th pass, we must figure out when a number has an odd number of factors.

Let's look at an example again. The 16th locker has an odd number of factors: 1, 2, 4, 8, 16. If we pair off the factors, we have (1, 16), (2, 8), and (4,4). In the first two pairs, the first factor toggles the locker open or closed and the second factor toggles the locker back to it's former position. That is because these are unique factors and thus different passes. But in the 3rd pair, the factors are the same. We know that we can only make the 4th pass once, so this factor is only included once in our factor list. Thus it toggles the locker open, but is not paired with another pass to close it again. If we consider the case when we have a non-unique pair of factors, we can realize that this occurs when the $n$th locker is a perfect square ( 4 * 4 = 16). We can write out the factors for any other perfect square and we will see that there is an odd number of factors. In this way, we can realize that the only lockers that will remain open are the locker numbers that are perfect squares. So, we have:

1*1 = 1, 2*2 = 4, 3*3 = 9, 4*4 = 16, 5*5 = 25, 6*6 = 36, 7*7 = 49, 8*8 = 64, 9*9 = 81, and 10*10 = 100. There are 10 perfect squares, and so after the 100th pass, 10 lockers will remain open.

\section*{Hints}
 
%Every hint needs to be preceded by a \item and between the \being{enumerate} and \end. Each \item will produce a seperated numbered item in an indented list. 

\begin{enumerate}
	\item Choose a random locker and consider the passes in which it is toggled. What about those passes stands out to you? How many passes affect that locker? 

	\item Consider what that tells you about when that locker will be opened or closed. What is the state of that locker after the final pass on it? How does that relate to when the locker is toggled?

	\item  Consider when the number of passes on the locker is even and when it is odd.

	\item When do we get an even number of passes on a locker?

\end{enumerate} 

\section*{Sources}

Question, answer and other material taken from Cracking the Coding Interview 6th edition by Gayle Laakmann McDowell.

\end{document}
